\documentclass{article}
\usepackage[english]{babel}
\usepackage{indentfirst}
\usepackage{float}
\usepackage{authblk}
\usepackage{amsmath}
\usepackage{graphicx}
\usepackage{blindtext}
\usepackage{cite}
%\usepackage{xr}
\setlength{\parindent}{0pt}

\title{Latex Exercise} %title
\author{Yusi Qin, Xiwei Wang, Jin Zhang \thanks{ABC}}
\affil{University of Zurich}
\date{\today}

\begin{document}
\maketitle
\begin{abstract}
This article is for Latex exercise. This article is for Latex exercise. This article is for Latex exercise.
\end{abstract}

\newpage
\section{dummy text}
\blindtext[8]

\newpage
\section{citation}
%\externaldocument{playground/myfirstline}
\cite{JKLVVV}

\cite{NJGRSJK}

\newpage
\section{Figure}

\begin{figure}[H]
\centering
\includegraphics[width=0.8\textwidth]{worldmap.png}
\caption{World Map}
\end{figure}

\newpage
\section{Table}

\begin{table}[H]
\centering
\begin{tabular}{|c|c|c|c|}
\hline
Table&title1&title2&title3\\ 
\hline
set1&conten1&conten2&conten3\\ 
\hline 
set2&conten1&conten2&conten3\\ 
\hline 
set3&conten1&conten2&conten3\\
\hline
\end{tabular}
\caption{Table Template}
\end{table}


\newpage
\bibliographystyle{unsrt}
\bibliography{references} 

\end{document}